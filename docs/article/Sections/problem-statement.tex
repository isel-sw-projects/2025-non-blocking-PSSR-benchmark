\section{Problem Statement}

In this section, we examine the challenges of implementing \textit{Progressive
Server-Side Rendering} (PSSR) in modern web applications, with a focus on the
limitations of current template engine designs. Our goal is to broaden the
range of options available for PSSR, particularly within JVM-based frameworks.
% Reactive streams~\cite{ReactiveStreams}, such as those based on
% \texttt{Observable} (from RxJava)~\cite{rx-observable} and \texttt{Flow} (from
% the Java 9+ standard library~\footnote{~\url{https://openjdk.org/jeps/266}}),
% are abstractions that provide a non-blocking, asynchronous
% way to handle data streams.
% These abstractions support the incremental generation and transmission of HTML
% content as data becomes available, enabling \textit{progressive server-side
% rendering} (PSSR)~\cite{pssr2005}.
In PSSR, the server does not
wait for the entire data model to be ready before beginning to render HTML.
Instead, it processes and streams each piece of data to the client as soon as
it arrives. 

Reactive types like \texttt{Observable<T>} of ReactiveX~\cite{rxjava} or Kotlin \texttt{Flow<T>}~\cite{kotlinlang}
facilitate this by representing data as a sequence of asynchronous events.
For example, a reactive stream might emit a sequence of \texttt{Presentation}
objects, each representing a talk in a conference schedule.
As each \texttt{Presentation} is emitted by the \texttt{Observable}
or \texttt{Flow}, the internal DSL-based engine—such as HtmlFlow—can render the
corresponding HTML fragment and immediately flush it to the client. This
approach is demonstrated in \autoref{lst:presentation-observable}, where each
presentation is rendered asynchronously as it is emitted by an
\texttt{Observable}.
Note that the \texttt{await} builder receives an additional parameter, the
\texttt{onCompletion} callback, which is used to signal HtmlFlow that it can
proceed to render the next HTML element in the web
template~\cite{carvalho2023async}.
HtmlFlow pauses the rendering process until \texttt{onCompletion} is called,
similar to how the \emph{resume} function works in continuations and
coroutines~\cite{coroutines_continuations}.
In \autoref{lst:presentation-flow}, we show an
equivalent suspend-based implementation using Kotlin's \texttt{Flow}~\cite{wise2024pssr}.
Both examples highlight how internal DSLs can natively integrate with reactive types
to enable non-blocking, progressive rendering on the server side.

\begin{center}
\begin{minipage}{0.50\textwidth}
\begin{lstlisting}[
    language=Kotlin,
    basicstyle=\scriptsize\ttfamily,
    numbers=none,
    caption={\textit{HtmlFlow reactive} presentation template in Koltin with an \texttt{Observable} model},
    label={lst:presentation-observable}
]
await { div, model, onCompletion ->
  model
    .doOnNext { presentation ->
      presentationFragmentAsync
        .renderAsync(presentation)
        .thenApply { frag -> div.raw(frag) }
      }
    .doOnComplete { onCompletion.finish() }
    .subscribe()
}
\end{lstlisting}
\end{minipage}
\hfill
\begin{minipage}{0.46\textwidth}
\begin{lstlisting}[
    language=Kotlin,
    basicstyle=\scriptsize\ttfamily,
    numbers=none,
    label={lst:presentation-flow},
    caption={\textit{HtmlFlow suspend} presentation template in Kotlin with a \texttt{Flow} model},
]
suspending { model ->
  model
    .toFlowable()
    .asFlow()
    .collect { presentation ->
      presentationFragmentAsync
        .renderAsync(presentation)
        .thenApply { frag -> raw(frag) }
    }
}
\end{lstlisting}
\end{minipage}
\end{center}

By contrast, template engines that use \textit{external} DSLs—such as JStachio,
Thymeleaf, or Handlebars—typically define templates within static HTML
documents using custom markers, and rely on blocking interfaces like
\texttt{java.util.Iterable} or \texttt{java.util.stream.Stream}.
These interfaces require the entire data
model, to be materialized in memory before rendering can begin, which blocks server
threads during template expansion and significantly limits scalability under
high concurrency.
Some reactive libraries, such as RxJava, provide bridging mechanisms like
\texttt{Observable.blockingIterable()}, which allows asynchronous data sources
to be exposed as \texttt{Iterable} by blocking the thread until all items are available.
While useful for compatibility with traditional APIs, this approach
reintroduces blocking behavior and undermines the benefits of non-blocking
I/O—especially under high concurrency. \autoref{lst:presentation-jstachio}
illustrates this model using a JStachio template, where the engine performs a
blocking loop over \texttt{presentationItems}.

\begin{center}
\begin{minipage}{0.70\textwidth}
\begin{lstlisting}[
  language=Kotlin,
  numbers=none,
  basicstyle=\scriptsize\ttfamily,
  caption={Presentation HTML template using \textit{JStachio}},
  label={lst:presentation-jstachio}
]
{{#presentationItems}}
<div class="card mb-3 shadow-sm rounded">
    <div class="card-header">
        <h5 class="card-title">
            {{title}} - {{speakerName}}
        </h5>
    </div>
    <div class="card-body">
        {{summary}}
    </div>
</div>
{{/presentationItems}}
\end{lstlisting}
\end{minipage}
\end{center}

Despite these performance limitations, external DSLs remain popular due to
several advantages:

\begin{enumerate}
    \item \emph{Separation of Concerns}: HTML templates are decoupled from application logic, enabling front-end developers to contribute without modifying back-end code.
    \item \emph{Cross-Language Compatibility}: External DSLs are portable across languages and frameworks, easing integration in multi-language environments.
    \item \emph{Familiarity}: Many developers are comfortable with HTML syntax, lowering the barrier to entry and improving maintainability.
\end{enumerate}

These strengths make external DSLs appealing—even when they come at the cost of
blocking, synchronous rendering. However, this tradeoff becomes critical under
high concurrency, where blocking threads severely degrades throughput~\cite{wise2024pssr}.
Emerging features in the Java ecosystem, particularly \textit{virtual threads}
introduced in Java 21 as part of Project Loom~\cite{Veen2024}, offer a promising solution to
this challenge. Virtual threads drastically reduce the overhead of blocking
operations by decoupling thread execution from OS-level threads. For this reason,
engines that rely on blocking interfaces—like those used in
external DSLs—can potentially achieve scalability levels that approach those of
non-blocking, asynchronous engines.

% Incluir este parágrafo?
% In previous work \cite{PSSR-WISE2024}, Carvalho benchmarked the performance of
% HtmlFlow (an internal DSL) using suspendable templates within a Spring WebFlux
% application. HtmlFlow scaled efficiently to 128 concurrent users and delivered
% up to 4,000 requests per second. In contrast, blocking engines such as JStachio
% saturated at just 4 concurrent users, maxing out at around 400 requests per
% second—highlighting the scalability limitations of blocking models.This work
% builds upon those findings by evaluating whether virtual threads can close this
% performance gap for blocking template engines. Specifically, we aim to
% determine whether external DSLs—traditionally unsuitable for PSSR due to their
% reliance on blocking APIs—can become viable for progressive rendering when
% powered by virtual threads.

% If successful, this approach would enable developers to reap the ergonomic
% benefits of external DSLs while maintaining scalable performance, without
% adopting complex asynchronous paradigms. It would also simplify the
% implementation of PSSR across a broader range of frameworks and engines, making
% the technique more accessible to mainstream web development.