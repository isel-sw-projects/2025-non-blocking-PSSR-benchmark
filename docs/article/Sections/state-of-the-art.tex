\section{State of the Art}

In this section, we first present the main properties that characterize each web
template technology approach, along with the advantages and drawbacks resulting
from these characteristics.

Next, we delve into the different design choices adopted by web servers in their
internal architectures, and how these choices impact the behavior of the web
template engines described in the previous subsection.

\subsection{Related Work}

\textit{Web templates} have been the most widely adopted approach for
constructing dynamic HTML pages.
Web templates (such as JSP, Handlebars, or Thymeleaf), also known as a
\textit{web pages} or \textit{web views}~\cite{Fowler02,Alur01}, are based on
HTML documents augmented with template-specific markers (e.g., \texttt{<\%>},
\texttt{\{\{\}\}}, or \texttt{\$\{\}}), which represent \textit{dynamic}
information to be replaced at runtime with the results of corresponding
computations, producing the final HTML page.
The process of parsing and replacing these markers---i.e.,
\textit{resolution}---is the primary responsibility of the \textit{template
engine}~\cite{Parr04}.

One key characteristic of \textit{web templates} is their ability to receive a 
\textit{context object}---equivalent to the \textit{model} in the model-view 
design pattern~\cite{mvc88,Parr04}---which provides the data used to fill 
template placeholders at runtime.

Web templates can be distinguished by several properties, namely:
\begin{enumerate}
    \item Templating idiom
    \item Supported data model APIs
    \item Asynchronous support
    \item Type safety and HTML safety
    \item Progressive rendering
\end{enumerate}

Although some of the aforementioned characteristics apply to both server-side 
and client-side approaches, we focus solely on web template technologies for 
server-side rendering, as our work is centered on that approach.

\subsubsection{Templating idiom}

\subsubsection{Supported data model APIs}

Restricted or unrestricted.

\subsubsection{Asynchronous support}

Explicar Observable e modelos reactivos

Distinguish between template engines that support and do not support asynchronous data models.

1. Nem se quer Existe API para iterar sobre um Async data model. 
   E.g JStachio, .... apenas suportam interface Iterable. => Compilation Error.
2. Consegue usar qq API e.g Kotlinx.Html and Groovy  => usa mal => produzir HTML out of order
3. Sim suporta => well-formed HTML => Thymleaf e o HtmlFlow

\subsubsection{Type safety and HTML safety}

\subsubsection{Progressive rendering}

%%%%%%%%%%%%%%%%%%%%%%%%%%%%%%%%%%%%%%%%%%%%%%%%%%%%%%%%%%%%%%%%%%%%
%-------------------------------------------------------------------
%%%%%%%%%%%%%%%%%%%%%%%%%%%%%%%%%%%%%%%%%%%%%%%%%%%%%%%%%%%%%%%%%%%%

\subsection{Web Framework Architectures and Approaches to PSSR}

\subsubsection{Spring WebFlux}

\subsubsection{Spring MVC}

\subsubsection{Quarkus}