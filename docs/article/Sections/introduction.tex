\section{Introduction}

% Non-blocking Progressive Server-Side Rendering (PSSR) is a technique that
% combines some of the advantages of both server-side and client-side rendering
% to improve web application performance and user experience. By leveraging
% non-blocking I/O operations, PSSR allows for efficient handling of multiple
% concurrent requests while progressively streaming HTML content to the client.
% 
% This project consists of a benchmark that evaluates the performance of
% different PSSR strategies on the Java Virtual Machine (JVM). The benchmark
% focuses on evaluating the performance of non-blocking PSSR implementations
% using approaches such as Kotlin coroutines, reactive-style programming, and
% Java virtual threads. Its main objective is to evaluate virtual threads as an
% alternative that ensures a non-blocking approach to PSSR while offering a more
% straightforward programming model than traditional asynchronous programming
% techniques.
% 
% This project aims to extend the previous work done in
% \textit{spring-webflux-comparing-template-engines}
% \citeauthor{spring-webflux-comparing-template-engines}, which compares the
% performance of various template engines in a Spring WebFlux application using
% different approaches to PSSR\@. It adds support for Java virtual threads and
% evaluates non-blocking PSSR implementations in other frameworks compared to
% Spring WebFlux.
% 
% Currently, this project features a working benchmark in Spring WebFlux, Spring
% MVC, and Quarkus, comparing the performance of eight different template
% engines: Rocker, JStachio, Pebble, Freemarker, Trimou, Thymeleaf, HtmlFlow, and
% KotlinX. It includes Java Microbenchmark Harness (JMH) benchmarks for template
% engine performance testing, as well as Apache Bench and Apache JMeter scripts
% for evaluating throughput and scalability. In the future, the project could be
% extended to include other frameworks such as Vert.x, if time allows. However,
% the next objective is the completion of the written report and article
% submission. The written report will include a more detailed analysis of the
% benchmark results, along with a contextualization of the state of the art in
% non-blocking PSSR and asynchronous programming.
% 
% This report will focus on detailing relevant aspects of the benchmark
% implementation and methodology, and will also provide a brief overview of the
% results obtained from the JMeter and Apache Bench tests, which evaluate the
% throughput and scalability of the different implementations.
